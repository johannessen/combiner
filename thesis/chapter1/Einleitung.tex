% UTF-8

% single-chapter commands
\documentclass[../main/thesis.tex]{subfiles}
\onlyinsubfile{\pagenumbering{roman}}
\begin{document}


\chapter{Einleitung}

%\section{Motivation und Zielsetzung}
%\subsection{Motivation}

Interaktive Kartendienste wie Google~Maps haben dazu beigetragen, Geoinformationen allgegenwärtig zu machen.
Jeder Mensch kann mit ihrer Hilfe im World~Wide~Web in Sekundenschnelle hochpräzise Karten von beinahe jedem Ort der Erde einsehen.

Die Qualität der kartographischen Darstellung leidet dabei jedoch in vielen Fällen.
% "good enough maps"? -> KN in AWA
% vgl. Tanja Pfeffer, https://lists.openstreetmap.org/pipermail/talk-de/2017-July/114166.html
So ist es unmöglich, die Masse an Informationen einer für solche Kartendienste genutzten weltweiten Geodatenbank manuell kartographisch zu generalisieren.
% "nicht praktisch möglich"?
% "G. is and will always be a subjective activity. It is difficult to set up fixed rules, although experiments to do so …" [KO03, 78]
%Die freie Wahl des Maßstabsbereichs in interaktiven Karten würde zudem erfordern, dass die Generalisierung mehrfach für unterschiedliche Zielmaßstäbe durchgeführt wird.

Dies gilt um so mehr für das Projekt OpenStreetMap~(OSM), das seine Geodaten vornehmlich aus Beiträgen Freiwilliger bezieht.
Vielen von ihnen fehlt eine kartographische Ausbildung.
Auch deshalb muss eine Generalisierung nach kartographischen Kriterien für \osm-Webkarten mit weltweiter Abdeckung automatisiert ablaufen.



%\subsection{Zielsetzung}

Bislang findet eine Generalisierung von Linienzügen in \osm-Karten nur in geringstem Umfang statt.
Ansätze für automatisierte Verfahren existieren zwar bereits, sie erfordern jedoch meist eine manuelle Nachbearbeitung und sind damit für \osm\ ungeeignet.

Ziel der vorliegenden Arbeit ist die Entwicklung von Algorithmen zur Generalisierung durch Zusammenfassung paralleler \osm-Linienzüge (etwa zwei Richtungsfahrbahnen einer Autobahn) auf eine gemeinsame Mittellinie.
Diese Vereinfachung der Geodaten soll die Qualität der kartographischen Darstellung verbessern und die eventuelle Weiterverarbeitung erleichtern.
% "soll dazu beitragen"?




%\section{Aufbau und Herangehensweise}
%\subsection{Vorgehensweise}
%\medskip

Die Richtungsfahrbahnen einer Autobahn sind nur eines von vielen Beispielen für parallele Linienzüge in der \osm-Datenbank.
Die Vielfalt der möglichen Situationen verbietet es, im Zeitrahmen einer Diplomarbeit eine Lösung mit allgemeiner Anwendbarkeit anzustreben.

Eine praxistaugliche Lösung für wenigstens eine solche Situation könnte jedoch bereits unmittelbare Verbesserungen bringen.
% der Darstellung und Weiterverarbeitung
Diese Arbeit beschränkt sich daher auf die Untersuchung eines einzelnen, noch festzulegenden Spezialfalls.
Dessen Lösung ließe sich dann möglicherweise auf andere Situationen übertragen.

Die zu entwickelnden Algorithmen sollen zum Nachweis ihrer Funktionsfähigkeit ausführbar implementiert werden.
Diese Implementierung könnte zugleich ein möglicher Ansatzpunkt für einen eventuellen späteren praktischen Einsatz des Verfahrens sein.
%, etwa zur Verbesserung der Darstellung von Webkarten.
Um dies nicht zu erschweren, soll die so entstehende Software unter einer freizügigen Open-Source-Lizenz verfügbar sein.

Maschinell ausführbare Beschreibungen sind notwendigerweise präzise bis ins letzte Detail, weswegen Software-Quelltext oft nicht leicht zu lesen ist.
Deshalb sollen die entwickelten Algorithmen in dieser Arbeit dem leichteren Verständnis zuliebe abstrakt beschrieben werden.
Weil dadurch implementierungsspezifische Details entfallen können, wird gleichzeitig die Weiterverwendung der entwickelten Ideen unabhängig von der Software ermöglicht.

Der Verzicht auf Detailgenauigkeit darf an dieser Stelle nicht mit Unbestimmtheit verwechselt werden:
„Being abstract is something profoundly different from being vague: by abstraction [...] one creates a new semantic level on which one can again be absolutely precise.“ \citex[1]{Dijk78}



%\subsection{Aufbau}

%Diese Arbeit ist wie folgt aufgebaut:
In dieser Arbeit umreißt Kapitel~\ref{ch:analysis} das Projekt \osm\ und gibt zahlreiche Kartenbeispiele dazu.
Diese vermitteln einen Eindruck davon, wie Liniengeneralisierung bisher angewandt wird und wo ihr Fehlen das Kartenbild verschlechtert.
Anschließend werden existierende Verfahren zur automatisierten Zusammenfassung von Linienzügen betrachtet und bewertet.

%Den Spezialfall, anhand dessen das Problem im Weiteren untersucht werden soll, legt Kapitel~\ref{ch:case} fest.
Kapitel~\ref{ch:case} legt den Spezialfall fest, anhand dessen das Problem im Weiteren untersucht werden soll.
Neben einer illustrierten Erläuterung des Funktionsprinzips enthält Kapitel~\ref{ch:algorithm} die formale und abstrakte Beschreibung der entwickelten Algorithmen;
Anhang~\ref{appx:mathsymbols} erklärt die dabei verwendeten Zeichen und Abkürzungen.
Kapitel~\ref{ch:impl} erläutert wesentliche Entscheidungen in Bezug auf die Implementierung in Software und beschreibt unerwartete Erkenntnisse.

Anschließend diskutiert Kapitel~\ref{ch:result} die Tauglichkeit der entwickelten Methode anhand einiger Beispiele.
Dabei wird auch die Anwendbarkeit auf andere als den zuvor festgelegten Spezialfall untersucht.
Schließlich gibt Kapitel~\ref{ch:conclusion} eine Gesamtbeurteilung über das Ergebnis dieser Arbeit ab und diskutiert, welche Ansätze zur Weiterentwicklung am lohnendsten erscheinen.
Kapitel~\ref{ch:summary} fasst Kontext, Vorgehen und Ergebnisse der Arbeit kurz zusammen.



\end{document}
