% UTF-8

% single-chapter commands
\documentclass[../main/thesis.tex]{subfiles}
\onlyinsubfile{\setcounter{chapter}{5}}  % single-chapter command
\begin{document}


\chapter{Ergebnisuntersuchung im Anwendungskontext}

\section{Beurteilungskriterien}

\begin{itemize}
	\item allgemeine Kriterien zur qualitativen Kritik des Ergebnisses (= des entwickelten Algorithmus / der entwickelten Software) beschreiben
	\item falls möglich, quantitative Metriken für die Beurteilung definieren
	\item erwartete Ergebnisse ausgehend von Analyse und Spezifikation rekapitulieren
\end{itemize}



\section{Ergebnisbeurteilung}

\begin{itemize}
	\item Beurteilung anhand der zuvor beschriebenen Kriterien
	\item Diskussion von signifikanten Abweichungen des beobachteten vom erwarteten Ergebnis
	\item Untersuchung der Anwendung des Algorithmus auf unterschiedliche Beispiele (unterschiedliche Regionen der Welt etc.) der spezifizierten Spezialfälle sowie auf andere als diese spezifizierten Fälle (highways statt railways etc.)
	\item abschließende quantitative Gesamtbeurteilung des Ergebnisses
\end{itemize}

% relocateGeneralisedNodes: "This is necessary because nodes are implemented as immutable by this project." -> Alternative: beide Nodes auf denselben Ort bewegen und dann als letzten Schritt Nodes am selben Ort erkennen und zusammenfassen (dann sollten aber sinnvollerweise auch die Node-IDs mitgeschleppt werden, sonst bringt das wenig, und die haben wir nicht, da wir OSM nicht direkt einlesen, sondern über die Geofabrik-Shapefiles gehen). Wichtig: *nur* wegen relocate() bekommt SourceNode die "edges" als Pointers!



% single-chapter commands
%\onlyinsubfile{\listoffigures}
%\onlyinsubfile{\listoftables}
%\onlyinsubfile{% global bibliography settings

\nocite{*}  % include works in bibliography that aren't cited anywhere in the document (for debugging)

\setbibpreamble{Die Literaturangaben sind alphabetisch nach den Nachnamen der Autoren sortiert. Bei mehreren Autoren wird nach dem ersten Autor sortiert.\par\bigskip\bigskip}

\bibliography{../references-papers,../references-manual}
%\bibliography{../references-manual}
}
\end{document}
