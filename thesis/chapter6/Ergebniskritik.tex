% UTF-8

% single-chapter commands
\documentclass[../main/thesis.tex]{subfiles}
\onlyinsubfile{\setcounter{chapter}{5}}  % single-chapter command
\begin{document}


\chapter{Ergebnisdiskussion}
% Beschreibung, was tatsächlich die Software ausspuckt, wo's hakt und wo ich noch gebastelt habe
% alles NUR im "Anwendungskontext", d.h. in Bezug auf die Ausgabe-Geodaten, nicht die Softwarequalität (das war in 5.4!)
% Schritt für Schritt Probleme sammeln und katalogisieren

Die in Kap. 4 und 5 entwickelte Software ist das Ergebnis dieser Arbeit. Hier diskutiert werden soll die Arbeitsweise dieser Software und die Qualität der von ihr ausgegebenen Geodaten in Bezug auf die Aufgabenstellung.

% möglichst auch Untersuchung der Anwendung des Algorithmus auf Straßen in unterschiedlichen Regionen der Welt



\section{„trivialer Fall“}

\begin{itemize}
\item Nachweis, dass dieser Ansatz grundsätzlich funktioniert; mit Screenshots uvm., auch von Signaturdefinition in QGIS
\item möglichst quantifizieren („bei Eingabedaten von xy km Autobahn gibt es z Problemsituationen“)
\item offen: Nicht-motorway/trunk sind schwieriger, da kleinere Kurvenradien. Funktioniert's hier? Wo nicht?
\item offen: Die \textproc{Distanz} zweier Straßen ist eines der wesentlichen Kriterien für die Erkennung als \textproc{Parallel}. Macht das Probleme, zB innerstädtisch vs. Autobahn?
\end{itemize}



\section{Attribute}

\begin{itemize}
\item (vorläufig) implementiertes Prinzip beschreiben
\item Auswirkungen: Beispiel, wo es nicht klappt?
\item möglichst quantifizieren
\end{itemize}



\section{Verhalten an Straßenkreuzungen}

\subsection{relocateGeneralisedNodes}

\begin{itemize}
\item tlws. gelöstes Problem: an Ende der Ausbaustrecke sowie innerstädtischer Kreuzung nodes von (nicht generalisierter) Querstraße so verschieben, dass es passt
\item offen: löst Problem nicht vollständig: funktioniert nicht (?), wenn Querstraße generalisiert ist?
\item (auch Problem im trivialen Fall, etwa in Heumar oder Kanalstraße/A57: Lücke in Topologie)
\item möglichst quantifizieren
\end{itemize}

% relocateGeneralisedNodes: "This is necessary because nodes are implemented as immutable by this project." -> Alternative: beide Nodes auf denselben Ort bewegen und dann als letzten Schritt Nodes am selben Ort erkennen und zusammenfassen (dann sollten aber sinnvollerweise auch die Node-IDs mitgeschleppt werden, sonst bringt das wenig, und die haben wir nicht, da wir OSM nicht direkt einlesen, sondern über die Geofabrik-Shapefiles gehen). Wichtig: *nur* wegen relocate() bekommt SourceNode die "edges" als Pointers!



\subsection{fehlende Kreuzungserkennung}

\begin{itemize}
\item fehlende Kreuzungserkennung
\item umfangreiche Problemdarstellung verschiedener Typen von Kreuzungen
\item möglichst quantifizieren
\end{itemize}



\section{Effizienz}

...



\section{Wiederholtes Ausführen für mehr als zwei Parallele}

...



\section{Anwendung auf andere Spezialfälle}

...



% single-chapter commands
%\onlyinsubfile{\listoffigures}
%\onlyinsubfile{\listoftables}
%\onlyinsubfile{% global bibliography settings

\nocite{*}  % include works in bibliography that aren't cited anywhere in the document (for debugging)

\setbibpreamble{Die Literaturangaben sind alphabetisch nach den Nachnamen der Autoren sortiert. Bei mehreren Autoren wird nach dem ersten Autor sortiert.\par\bigskip\bigskip}

\bibliography{../references-papers,../references-manual}
%\bibliography{../references-manual}
}
\end{document}
