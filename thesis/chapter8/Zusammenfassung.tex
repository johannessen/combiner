% UTF-8

% single-chapter commands
\documentclass[../main/thesis.tex]{subfiles}
\onlyinsubfile{\setcounter{chapter}{7}}  % single-chapter command
\begin{document}


\chapter{Zusammenfassung}

Das Projekt OpenStreetMap (OSM) hat das Ziel der Erstellung einer freien Geodatenbank auf Basis von \term{volunteered geographic information} (VGI).
Die weitere Verarbeitung und Visualisierung von \osm-Daten läuft in aller Regel voll automatisiert ab.
Sie wird erschwert durch den teilweise sehr hohen Detailreichtum, die daraus folgende Fragmentierung von Linienzügen sowie unvollständige Verknüpfungen zusammenhängender Geodaten wie etwa parallelen Richtungsfahrbahnen über Relationen im Datenmodell.

Eine kartographische Generalisierung von \osm-Daten findet bisher nur in geringstem Umfang statt.
Dies fällt unter anderem bei parallelen Richtungsfahrbahnen auf, deren Straßenachse in \osm\ nicht erfasst ist und bisher auch nicht in zufriedenstellender Weise automatisiert abgeleitet werden kann.
%Hier kommt es verbreitet zu unbefriedigenden Darstellungen wie etwa dem Unterschreiten kartographischer Mindestgrößen sowie Fehlern wie etwa dem Überkreuzen der beiden Richtungsfahrbahnen bei Formvereinfachung.
Ansätze zur automatisierten Zusammenfassung von Linienzügen existieren, sind jedoch auf \osm-Daten nicht gut anwendbar.
Insbesondere können sie Kreuzungssituationen oft nicht ohne besondere Attribute lösen.

Diese Arbeit stellt eine Methode zur Erkennung paralleler Linienzüge auf der Basis eines geometrischen Vergleichs kurzer Fragmente vor.
Linienzüge aus \osm\ werden so lange unterteilt, bis sich Stützpunkte auf Parallelen derart einander gegenüberliegen, dass eine Prüfung auf Parallelität leicht möglich ist.
Die anschließende Zusammenfassung der erkannten Parallelen ist dann einfach zu lösen.
Der Rechenaufwand der entwickelten Algorithmen wächst linear mit der Anzahl der Stützpunkte ($\mathcal{O}(n)$).
% eigentlich linear zur Anzahl der Segmente, aber deren Zahl ist proportional zur Anzahl der Stützpunkte, so dass dies keinen Unterschied macht

Zum Nachweis ihrer Funktionsfähigkeit und zum Test mit \term{real world}--Daten aus \osm\ erfolgte ihre ausführbare Implementierung.
Aufgrund einiger technischer Schwierigkeiten war dies aufwändiger als erwartet.
% was den Fokus ein Stück weit weg von der Kartographie hin zur Informatik verschob
Die mit Java entwickelte Software („Combiner“) hat erhebliches Optimierungspotenzial.

Wie sich zeigt, führt die mit dieser Arbeit entwickelte Methode in vielen Fällen zu einem guten Generalisierungsergebnis.
Jedoch leidet auch diese Methode an erheblichen Problemen in Kreuzungssituationen.
Aus Zeitgründen war es nicht möglich, eine praxistaugliche Lösung für diese Probleme zu finden.

% Attribute könnten hier noch erwähnt werden ... sie waren zwar bisher kein großer Teil der Arbeit, müssten es aber werden, wenn Praxistauglichkeit erreicht werden soll

Auch für andere, parallel entwickelte Methoden neueren Datums wird von ähnlichen Problemen in Kreuzungssituationen berichtet.
Eine offensichtliche Lösung mit allgemeiner Anwendbarkeit für das Problem der Zusammenfassung paralleler Linienzüge zeichnet sich derzeit nicht ab.
Es ist jedoch anzunehmen, dass eine zuverlässige automatisierte Kreuzungserkennung die Zusammenfassung zu einem leicht lösbaren Problem machen würde.
Diese Arbeit benennt dazu mehrere unterschiedliche mögliche Ansätze.



\chapter*{Summary}
\addcontentsline{toc}{chapter}{Summary}

...



% single-chapter commands
\end{document}
