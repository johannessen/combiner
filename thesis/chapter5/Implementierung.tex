% UTF-8

% single-chapter commands
\documentclass[../main/thesis.tex]{subfiles}
\onlyinsubfile{\setcounter{chapter}{4}}  % single-chapter command
\begin{document}


\chapter{Implementierung}

\section{Entwicklungsumgebung}

Zur Umsetzung der entwickelten Algorithmen in Software wurde die Plattform Java verwendet.
Die Wahl von Java erfolgte neben der Vertrautheit des Verfassers mit dem zugehörigen \term{framework} aufgrund zu erwartender Effizienzvorteile von kompiliertem Code gegenüber Skriptsprachen wie Perl.
% einer Empfehlung der Geofabrik folgend

Java als imperative Sprache erlaubt es nicht, Algorithmen mit dem gleichen Grad an Abstraktion zu beschreiben wie zuvor in Kapitel~\ref{ch:algorithm-parts} geschehen.
Dort konnten zugunsten einer vereinfachten
% jedoch präzisen, cf. EWD656
Beschreibung praktische Erwägungen wie der Bedarf an Rechenzeit und Speicherplatz teilweise hintenanstehen.
Bei der Implementierung in Java müssen hingegen sorgfältig solche Datenstrukturen gewählt werden, die eine effiziente Ausführung erlauben, und die Algorithmen soweit nötig entsprechend angepasst werden.
Das Ergebnis wird in Abschnitt~\ref{ch:data-structures} beschrieben.

Während der Entwicklung wurde versucht, so viel existierenden Code in Form von \term{frameworks} wiederzuverwenden wie möglich.
Diese Bestrebung verursachte Probleme, wie auch später in Abschnitt~\ref{ch:impl-difficulties} geschildert wird.
Bei der Entwicklung kamen zuletzt die folgenden Plattformen und \term{frameworks} zum Einsatz:

\begin{itemize}[nosep]
	\item Darwin 15.6 / Mac OS X 10.11.6
	\item Java™ Standard Edition JDK 8 Update 144\\ \url{http://www.oracle.com/technetwork/pt/java/javase/downloads/}
	\item Apache Ant 1.10.1 \quad \url{https://ant.apache.org/}
	\item args4j 2.33 \quad \url{http://args4j.kohsuke.org/}
	\item GeoTools 17.0 \quad \url{http://www.geotools.org/}
	% as of 2017-09-14, 17.1 is stable and includes one or two possibly useful bugfixes; 17.2 is not stable and can prolly be skipped
	\item TestNG 6.8 \quad \url{http://testng.org/}
	% http://web.archive.org/web/20121113133417/http://testng.org/testng-6.8.zip (all other releases appear to be corrupt; I'm probably doing something wrong)
	\item GDAL 2.2.1 \quad \url{http://www.gdal.org/}
\end{itemize}

Ursprünglich wurden ältere Softwareversionen verwendet.
Die nötigen Anpassungen an die hier genannten aktuellen Versionen waren gering, Kompatibilität mit den älteren Versionen ist jedoch gegenwärtig aufgrund von Änderungen in GeoTools nicht mehr gegeben.
Der Code ist dabei noch immer konform zum Syntax von Java 6. \cf{GJSB05}



\section{Systemkonzept}

\begin{itemize}
	\item konzeptueller Überblick der entwickelten Software in dem Umfang, der für den Leser dieser Arbeit zum Verständnis notwendig ist
	\begin{itemize}
		\item z. B. Modulabhängigkeiten
		\item z. B. Klassenstrukturen
		\item z. B. Interaktionswege
	\end{itemize}
	\item Bezug zu Kapitel 4 herstellen
\end{itemize}



\section{Datenstrukturen}
\label{ch:data-structures}



\section{Schwierigkeiten bei der Umsetzung}
\label{ch:impl-difficulties}

\begin{itemize}
	\item Erläuterung wichtiger Designentscheidungen
	\item Unterschiede zu den entworfenen Algorithmen
	\item interessante konkrete Schwierigkeiten oder Erfolge bei der Implementierung aufzeigen
\end{itemize}


% single-chapter commands
%\onlyinsubfile{\listoffigures}
%\onlyinsubfile{\listoftables}
%\onlyinsubfile{% global bibliography settings

\nocite{*}  % include works in bibliography that aren't cited anywhere in the document (for debugging)

\setbibpreamble{Die Literaturangaben sind alphabetisch nach den Nachnamen der Autoren sortiert. Bei mehreren Autoren wird nach dem ersten Autor sortiert.\par\bigskip\bigskip}

\bibliography{../references-papers,../references-manual}
%\bibliography{../references-manual}
}
\end{document}
