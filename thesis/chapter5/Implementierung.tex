% UTF-8

% single-chapter commands
\documentclass[../main/thesis.tex]{subfiles}
\onlyinsubfile{\setcounter{chapter}{4}}  % single-chapter command
\begin{document}


\chapter{Implementierung}

\section{Entwicklungsumgebung}

Zur Umsetzung der entwickelten Algorithmen in Software wurde die Plattform Java verwendet.
Die Wahl von Java erfolgte neben der Vertrautheit des Verfassers mit dem zugehörigen \term{framework} aufgrund zu erwartender Effizienzvorteile von kompiliertem Code gegenüber Skriptsprachen wie Perl.
% einer Empfehlung der Geofabrik folgend

Java als imperative Sprache erlaubt es nicht, Algorithmen mit dem gleichen Grad an Abstraktion zu beschreiben wie zuvor in Kapitel~\ref{ch:algorithm-parts} geschehen.
Dort konnten zugunsten einer vereinfachten
% jedoch präzisen, cf. EWD656
Beschreibung praktische Erwägungen wie der Bedarf an Rechenzeit und Speicherplatz teilweise hintenanstehen.
Bei der Implementierung in Java müssen hingegen sorgfältig solche Datenstrukturen gewählt werden, die eine effiziente Ausführung erlauben, und die Algorithmen soweit nötig entsprechend angepasst werden.
Das Ergebnis wird in Abschnitt~\ref{ch:data-structures} beschrieben.

Während der Entwicklung wurde versucht, so viel existierenden Code in Form von \term{frameworks} wiederzuverwenden wie möglich.
Diese Bestrebung verursachte Probleme, wie auch später in Abschnitt~\ref{ch:impl-difficulties} geschildert wird.
Bei der Entwicklung kamen zuletzt die folgenden Plattformen und \term{frameworks} zum Einsatz:

\begin{itemize}[nosep]
	\item Darwin 15.6 / Mac OS X 10.11.6
	\item Java™ Standard Edition JDK 8 Update 144\\ \url{http://www.oracle.com/technetwork/pt/java/javase/downloads/}
	\item Apache Ant 1.10.1 \quad \url{https://ant.apache.org/}
	\item args4j 2.33 \quad \url{http://args4j.kohsuke.org/}
	\item GeoTools 17.0 \quad \url{http://www.geotools.org/}
	% as of 2017-09-14, 17.1 is stable and includes one or two possibly useful bugfixes; 17.2 is not stable and can prolly be skipped
	\item TestNG 6.8 \quad \url{http://testng.org/}
	% http://web.archive.org/web/20121113133417/http://testng.org/testng-6.8.zip (all other releases appear to be corrupt; I'm probably doing something wrong)
	\item GDAL 2.2.1 \quad \url{http://www.gdal.org/}
\end{itemize}

Ursprünglich wurden ältere Softwareversionen verwendet.
Die nötigen Anpassungen an die hier genannten aktuellen Versionen waren gering, Kompatibilität mit den älteren Versionen ist jedoch gegenwärtig aufgrund von Änderungen in GeoTools nicht mehr gegeben.
Der Code ist dabei noch immer konform zum Syntax von Java 6. \cf{GJSB05}



\section{Systemkonzept}

Für ein gut funktionierendes Gesamtpaket sind vor der puren Umsetzung von vorgegebenen Algorithmen in ausführbarem Code einige praktische Aspekte zu bedenken.

Zunächst stellt sich die Frage nach der Benutzerschnittstelle der Software.
Die gewählte Plattform Java bietet verschiedene Möglichkeiten, graphische Benutzeroberflächen (GUI) zu gestalten.
Denkbar wäre beispielsweise eine Integration in den weit verbreiteten \osm-Editor JOSM als \term{plug-in.}
Das Entwickeln und Debuggen einer GUI-Anwendung neigt jedoch dazu, zeitaufwändig zu sein.
Ohnehin würde es ein modularer Aufbau der Anwendung erlauben, zu einem späteren Zeitpunkt eine GUI zu ergänzen.
Aus diesen Gründen soll im Rahmen dieser Arbeit auf eine GUI zugunsten einer einfachen Kommandozeilen-Schnittstelle (CLI) verzichtet werden.

Zur Verarbeitung beliebiger Geodaten müssen diese aus Datenspeichern eingelesen und wieder ausgegeben werden können.
Um Entwicklungszeit zu sparen, sollte hierfür nach Möglichkeit ein existierendes \term{framework} genutzt werden.
Zur Vermeidung einer zu engen Kopplung der Generalisierung an das gewählte \term{framework} ist es zu vermeiden, die Primitiven des \term{frameworks} intern weiterzubenutzen.
Stattdessen sollten eigene Datenstrukturen genutzt werden, um Modularität zu fördern.
Die dabei entstehenden Kosten in Form von Rechenzeit und Speicherverbrauch sind zu berücksichtigen.

Da die Algorithmen auf der euklidischen Ebene definiert sind (vgl. Abschnitt~\ref{ch:split-algorithm}), bietet es sich an, die Implementierung zunächst auf kartesische Koordinaten zu beschränken und zu verlangen, dass aus der \osm-Datenbank stammende Eingabedaten nach Länge und Breite vor der Verarbeitung vom Benutzer in einem geeigneten Kartennetzentwurf abgebildet werden.
Grundsätzlich wären angesichts der hier durchzuführenden geometrischen Operationen winkeltreue Abbildungen wie etwa ein Mercator-Netz zu bevorzugen.
% Sny87, USGS PP 1395 p. 20
Aufgrund der üblicherweise geringen Abstände der Parallelen ist Winkeltreue jedoch kein strenges Kriterium.

\begin{itemize}
	\item konzeptueller Überblick der entwickelten Software in dem Umfang, der für den Leser dieser Arbeit zum Verständnis notwendig ist
	\begin{itemize}
		\item z. B. Modulabhängigkeiten
		\item z. B. Klassenstrukturen
		\item z. B. Interaktionswege
	\end{itemize}
	\item Bezug zu Kapitel 4 herstellen
\end{itemize}



\section{Datenstrukturen}
\label{ch:data-structures}



\section{Schwierigkeiten bei der Umsetzung}
\label{ch:impl-difficulties}

\begin{itemize}
	\item Erläuterung wichtiger Designentscheidungen
	\item Unterschiede zu den entworfenen Algorithmen
	\item interessante konkrete Schwierigkeiten oder Erfolge bei der Implementierung aufzeigen
\end{itemize}


% single-chapter commands
%\onlyinsubfile{\listoffigures}
%\onlyinsubfile{\listoftables}
%\onlyinsubfile{% global bibliography settings

\nocite{*}  % include works in bibliography that aren't cited anywhere in the document (for debugging)

\setbibpreamble{Die Literaturangaben sind alphabetisch nach den Nachnamen der Autoren sortiert. Bei mehreren Autoren wird nach dem ersten Autor sortiert.\par\bigskip\bigskip}

\bibliography{../references-papers,../references-manual}
%\bibliography{../references-manual}
}
\end{document}
