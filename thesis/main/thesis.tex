
\documentclass{../thesis}

\subtitle{Diplomarbeit}
\title{Algorithmen zur automatisierten Generalisierung durch Zusammenfassung von Linienzügen in OpenStreetMap\\ für konkrete Spezialfälle}
\author{Arne Johannessen}
\publishers{betreut durch\\ Prof. Dr. rer. nat. Detlef Günther-Diringer\\ und\\ Dipl.-Wi.-Ing. Frederik Ramm}
%\date{Arbeitsentwurf}
%\dedication{}

%\ccbysa
% https://creativecommons.org/licenses/by-sa/4.0/legalcode.de
% https://wiki.creativecommons.org/wiki/License_Versions
% ODBL: 2009 <http://www.ifross.org/artikel/open-database-license-odbl-veroeffentlicht>
\usepackage[type={CC},modifier={by-nd},version={4.0}]{doclicense}
\newcommand{\makelicensepage}{
\clearpage\thispagestyle{empty}\vspace*{\stretch{1}}
{\small

\noindent\copyright\ 2018 Arne~Johannessen, Detlef~Günther-Diringer, Frederik~Ramm
% https://de.wikipedia.org/wiki/Akademischer_Grad#Nennungspflicht_akademischer_Grade
%\noindent\begin{tabular}{@{}l@{}l@{}}
%\copyright\ 2018~ & Arne~Johannessen, Prof.\,Dr.\,rer.\,nat.\,Detlef~Günther-Diringer, \\
%& Dipl.-Wi.-Ing.\,Frederik~Ramm
%\end{tabular}
\vspace{3ex}

{\noindent
	\begin{minipage}{10em}
		\doclicenseImage%
	\end{minipage}
	\hspace{1em}
	\begin{minipage}{\linewidth-.5em-11em}
		Dieses Werk darf weitergegeben werden unter den Bedingungen der Lizenz „Creative Commons Namensnennung -- Keine Bearbeitungen 4.0 International“ (CC-BY-ND). \cf{license:cc-by-nd-4.0}
	\end{minipage}
}
\vspace{2ex}

\noindent Bestimmte Teile dieses Werks dürfen alternativ auch nach den
Bedingungen folgender anderer Lizenzen weiterverwendet werden:

\begin{description}[labelwidth=3.5em,leftmargin=4em,itemsep=1.5ex]

\item[\ccbysa]
Der gesamte \textbf{Text} oder Auszüge davon sowie alle \textbf{Abbildungen}
dürfen nach den freizügigeren Bedingungen der Lizenz „Creative Commons
Namensnennung -- Weitergabe unter gleichen Bedingungen 4.0 International“
(CC-BY-SA) \cf{license:cc-by-sa-4.0} bearbeitet und weitergegeben werden.
Davon \textbf{ausgenommen} sind diejenigen Abbildungen, welche in der
Bildunterschrift und im Abbildungsverzeichnis mit dem Symbol „\copyright“
gekennzeichnet sind. Bei letzteren handelt es sich um nach \S~51~UrhG
zulässige Großzitate aus geschützten Werken.
% Weil diese Abbildungen nicht unter CC-BY-SA weiterverwendet werden können,
% steht dieses Dokument in seiner Gesamtheit als PDF sowie in gedruckter
% Fassung unter CC-BY-ND. Dies beschränkt nicht die Wiederverwendbarkeit von
% denjenigen Teilen, die unter CC-BY-SA stehen.
% http://resikom.adw-goettingen.gwdg.de/Henschen.pdf

\item[\ccAttribution\kern0.1em\ccShareAlike~\footnotesize\textsf{ODbL}]
Dieses Werk enthält \textbf{Geodaten} aus OpenStreetMap, die hier unter den
Bedingungen der Open Database License (ODbL) \cf{license:odbl} verfügbar
gemacht werden. \copyright~OpenStreetMap-Mitwirkende.
% "Contains information from DATABASE NAME, which is made available here under the Open Database License (ODbL)."

\item[\ccAttribution~\footnotesize\textsf{BSD}]
Der \textbf{Quelltext} der im Rahmen dieser Arbeit entwickelten Software
mitsamt seiner Dokumentation darf unter den Bedingungen einer
3-Klausel-BSD-Lizenz („modifizierte“ BSD-Lizenz) verändert und weitergegeben
werden.

\end{description}

\noindent Soweit zulässig sind alle Bestandteile öffentlich zugänglich über GitHub:\\
\url{https://github.com/johannessen/combiner}


}\clearpage
}

% je4
%\pdfinfo{
%  /Author (Name)
%  /Title (Titel der Diplomarbeit)
%  /Producer     (pdfeTex 3.14159-1.30.6-2.2)
%  /Keywords ()
%}
%\hypersetup{
%pdftitle=Titel der Diplomarbeit,
%pdfauthor=Name,
%pdfsubject={Diplomarbeit},
%pdfproducer={pdfeTex 3.14159-1.30.6-2.2},
%colorlinks=false,
%%pdfborder=0 0 0	% keine Box um die Links!
%}

\begin{document}

\maketitle

%\begin{abstract}
%\end{abstract}
\clearpage{~}
\makelicensepage

\subfile{../main/AufgabeVorbemerkung}

\tableofcontents

\newpage\addcontentsline{toc}{chapter}{Abbildungsverzeichnis\medskip}
\listoffigures

\newpage\addcontentsline{toc}{chapter}{Tabellenverzeichnis\medskip}
\listoftables

%\listofalgorithms
% unified list: see KOMAscript 135

\renewcommand{\onlyinsubfile}[1]{}
%\renewcommand{\notinsubfile}[1]{#1}
\renewcommand{\BibTeXenabled}{1}

\subfile{../chapter1/Einleitung}
\subfile{../chapter2/Analyse}
\subfile{../chapter3/Spezifikation}
\subfile{../chapter4/Algorithmen}
\subfile{../chapter5/Implementierung}
\subfile{../chapter6/Ergebniskritik}
\subfile{../chapter7/Schlussfolgerung}
\subfile{../chapter8/Zusammenfassung}

\appendix
\addtocontents{toc}{\bigskip\medskip\textbf{Anhang}\par}
\subfile{../appendices/Anhang}

%\addtocontents{toc}{\medskip}
%\newpage\addcontentsline{toc}{chapter}{Summary}

% UTF-8

\documentclass[../main/thesis.tex]{subfiles}
\begin{document}

% include works in bibliography that aren't cited anywhere in the document (for debugging)
\onlyinsubfile{\nocite{*}}


\defbibnote{thesisBibIntro}{\justify%
Die Literaturangaben sind alphabetisch nach dem Kürzel sortiert.
Das Kürzel wird gebildet aus den ersten drei Buchstaben des Nachnamens des Autors, bei mehreren Autoren aus jeweils den Anfangsbuchstaben der Nachnamen, bei Körperschaften aus einer mnemonisch gewählten Folge von Kleinbuchstaben; jeweils ergänzt durch die letzten beiden Ziffern des Jahres der Veröffentlichung.
\par
Um ein eventuelles Nachschlagen zu erleichtern, sind die Referenzen wo immer möglich durch Angabe von Orten ergänzt, an denen eine Kopie des jeweiligen Werks am 1.~März 2018
% gegen 22~Uhr
aufzufinden war.
In der PDF-Ausgabe dieses Dokuments sind die URLs Hyperlinks.
Die Signaturen beziehen sich auf die Bibliothek des Karlsruher Instituts für Technologie und deren Standort „Fachbibliothek HsKA“.
\bigskip}


\RaggedRight
\addtocontents{toc}{\medskip}
\newpage\addcontentsline{toc}{chapter}{Literaturverzeichnis}
\printbibliography[title=Literaturverzeichnis,prenote=thesisBibIntro]

\end{document}


\end{document}
