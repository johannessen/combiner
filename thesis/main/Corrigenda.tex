% UTF-8

\documentclass[a5paper,draft,headings=small,DIV=16]{scrreprt}
\XeTeXinputnormalization=1
\usepackage[ngerman]{babel}
\selectlanguage{ngerman}
\setkomafont{sectioning}{\normalfont\bfseries}
\newcommand{\errataentrysep}{\vspace{2ex}}
\newcommand{\errataparsep}{\vspace{1ex}}

\newcommand{\BibLaTex}{\textsc{Bib}\LaTeX}
\newcommand{\reasonmincrossrefs}{
Eine falsche Einstellung für \texttt{\small mincrossrefs} in \BibLaTex\
führte zum Erscheinen dieser Einträge im Verzeichnis, obwohl sie im Text
nicht direkt verwendet wurden.}


\begin{document}


\chapter*{Corrigenda}
\thispagestyle{empty}
\begin{tabular}{@{}lp{9.9cm}@{}}


\multicolumn{2}{@{}l}{Seite~13, Absatz~4, Satz~3:} \\
\textbf{Ersetze} & „lohnensten“ durch „lohnendsten“. \errataentrysep\\


\multicolumn{2}{@{}l}{Seite~18, Zeile~1, „gezeichnet“:} \\
\textbf{Füge ein} & Fußnote: \footnotesize{„Im November 2015 wurden Anliegerstraßen auf Zoom~13 auf 2,5\,px verschmälert. Die Abbildungen sind neueren Datums.“} \errataentrysep\\


\multicolumn{2}{@{}l}{Seite~25, Abschnitt~2.4, Absatz~3, Satz~2:} \\
\textbf{Ersetze} & \textit{„passenger\_lines“} durch \textit{„tracks“.} \errataentrysep\\


\multicolumn{2}{@{}l}{Seite~50, Abschnitt~5.3.2, Satz~4:} \\
\textbf{Ersetze} & „Segmente“ durch „\textsc{Segmente}“. \\
\textbf{Ersetze} & „\textsc{Segmentierung}“ durch „Segmentierung“. \errataentrysep\\


\multicolumn{2}{@{}l}{Seite~51, Abbildung~43:} \\
\textbf{Ersetze} & „NonexistantNode“ durch „NonexistentNode“. \errataentrysep\\


\multicolumn{2}{@{}l}{Seite~58, Satz~2:} \\
\textbf{Ersetze} & „\textsc{Segmentierung}“ durch „Segmentierung“. \\
\textbf{Ersetze} & „Segmente“ durch „\textsc{Segmente}“. \errataentrysep\\


\multicolumn{2}{@{}l}{Seite~81, Abbildung~61:} \\
\textbf{Ersetze} & „NonexistantNode“ durch „NonexistentNode“. \errataentrysep\\


\multicolumn{2}{@{}l}{Seiten~85 bzw.~89:} \\
\textbf{Streiche} & Einträge [ica05] und [RKE06] im Literaturverzeichnis. \errataparsep

\reasonmincrossrefs
\errataentrysep\\


\end{tabular}

\errataentrysep
\noindent Stand: 5.~März 2018

\end{document}
