% UTF-8

\documentclass[a5paper,twoside,final,headings=small,fontsize=10.25pt,DIV=16,BCOR=3cm]{scrreprt}
\XeTeXinputnormalization=1
\usepackage[ngerman]{babel}
\selectlanguage{ngerman}
\setkomafont{sectioning}{\normalfont\bfseries}
\usepackage[raggedrightboxes]{ragged2e}
\usepackage{longtable}
\newcommand{\errataentrysep}{\vspace{2ex}}
\newcommand{\errataparsep}{\vspace{1ex}}

\newcommand{\BibLaTex}{\textsc{Bib}\LaTeX}
\newcommand{\reasonmincrossrefs}{
Eine falsche Einstellung für \texttt{\small mincrossrefs} in \BibLaTex\
führte zum Erscheinen dieser Einträge im Verzeichnis, obwohl sie im Text
nicht direkt verwendet wurden.}


\begin{document}


\chapter*{Corrigenda}
\thispagestyle{empty}\pagestyle{empty}
\begin{longtable}[l]{@{}lp{7.6cm}@{}}


\multicolumn{2}{@{}l}{Seite~13, Absatz~4, Satz~3:} \\
\textbf{Ersetze} & „lohnensten“ durch „lohnendsten“. \errataentrysep\\


\multicolumn{2}{@{}l}{Seite~18, Zeile~1, nach „gezeichnet“:} \\*
\textbf{Füge ein} & Fußnote: \footnotesize{„Im November 2015 wurden
Anliegerstraßen auf Zoom~13 auf 2,5\,px verschmälert. Die Abbildungen
sind neueren Datums.“} \errataentrysep\\


\multicolumn{2}{@{}l}{Seite~25, Abschnitt~2.4, Absatz~3, Satz~2:} \\
\textbf{Ersetze} & \textit{„passenger\_lines“} durch \textit{„tracks“.} \errataentrysep\\


\multicolumn{2}{@{}l}{Seite~26, Abschnitt~2.5.2, Absatz~2, Satz~2:} \\*
\textbf{Ersetze} & „[EE99]“ durch „[cf.~EE99]“. \errataentrysep\\


\multicolumn{2}{@{}l}{Seite~27, Zeile~4:} \\*
\textbf{Ersetze} & „([ME13], cf. [Mig12])“ \newline
durch „[cf.~Mig12; ME13]“. \errataentrysep\\


\multicolumn{2}{@{}l}{Seite~50, Abschnitt~5.3.2, Satz~4:} \\
\textbf{Ersetze} & „Segmente“ durch „\textsc{Segmente}“. \\
\textbf{Ersetze} & „\textsc{Segmentierung}“ durch „Segmentierung“. \errataentrysep\\


\multicolumn{2}{@{}l}{Seite~51, Abbildung~43:} \\
\textbf{Ersetze} & „NonexistantNode“ durch „NonexistentNode“. \errataentrysep\\


\multicolumn{2}{@{}l}{Seite~53, Absatz~1, Satz~5:} \\*
\textbf{Ersetze} & „\textsc{Segmentieren}“ durch „Segmentieren“. \errataentrysep\\


\multicolumn{2}{@{}l}{Seite~55, Abschnitt~5.3.5, Sätze~2--3:} \\*
\textbf{Ersetze} & „einen möglichst langen Linienzug“ \newline
durch „einzelne Segmente“. \\*
\textbf{Ersetze} & „hier“ durch „sowohl diese als“.\errataentrysep\\


\multicolumn{2}{@{}l}{Seite~58, Satz~2:} \\
\textbf{Ersetze} & „\textsc{Segmentierung}“ durch „Segmentierung“. \\
\textbf{Ersetze} & „Segmente“ durch „\textsc{Segmente}“. \errataentrysep\\


\multicolumn{2}{@{}l}{Seite~63, Abbildung~52:} \\*
\textbf{Streiche} & „cf.“ \errataentrysep\\


\multicolumn{2}{@{}l}{Seite~66, Abschnitt~6.5, Satz~1:} \\*
\textbf{Ersetze} & „zusammenzufassen“ durch „zusammenfassen“. \errataentrysep\\


\multicolumn{2}{@{}l}{Seite~77, Absatz~2, Satz~4, nach „particular“:} \\*
\textbf{Füge ein} & Komma. \errataentrysep\\


\multicolumn{2}{@{}l}{Seite~77, Absatz~4, Satz~2, nach „difficulties“:} \\*
\textbf{Füge ein} & Komma. \errataentrysep\\


\multicolumn{2}{@{}l}{Seite~81, Abbildung~61:} \\
\textbf{Ersetze} & „NonexistantNode“ durch „NonexistentNode“. \errataentrysep\\


\multicolumn{2}{@{}l}{Seiten~85 bzw.~89:} \\
\textbf{Streiche} & Einträge [ica05] und [RKE06] im Literaturverzeichnis. \errataparsep

\reasonmincrossrefs
\errataentrysep\\


\multicolumn{2}{@{}l}{Seiten~87--88, Einträge [NZ12] und [NZZ12]:} \\*
\textbf{Streiche} & „English.“ \errataentrysep\\


\multicolumn{2}{@{}l}{Seite~89, Eintrag [Sch12], nach „Nachrichten“:} \\*
\textbf{Füge ein} & „Jg. 62,“. \errataentrysep\\


\end{longtable}

\errataentrysep
\noindent Stand: 25.~März 2018

\end{document}
