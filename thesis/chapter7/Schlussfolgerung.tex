% UTF-8

% single-chapter commands
\documentclass[../main/thesis.tex]{subfiles}
\onlyinsubfile{\setcounter{chapter}{6}}  % single-chapter command
\begin{document}


\chapter{Schlussfolgerung und Ausblick}

\section{Praktische Anwendbarkeit \emph{[ausstehend]}}

Es konnte durch die Implementierung in Software gezeigt werden, dass die Erkennung paralleler Linienzüge auf der Basis eines geometrischen Vergleichs möglichst kurzer Fragmente prinzipiell funktioniert.
Mit den im Rahmen hierfür entwickelten Algorithmen können auch viele schwierige Situationen des in Abschnitt~\ref{3.1} ausgewählten Spezialfalls der „Richtungsfahrbahnen ...“ zufriedenstellend gelöst werden.

Bei der Anwendung auf komplexe Straßennetze zeigt sich jedoch, dass erhebliche Probleme im Kreuzungsbereich, wie sie bereits in Abschnitt~\ref{2.5} von anderen Ansätzen ähnlich berichtet wurden, auch bei den in dieser Arbeit vorgestellten Algorithmen auftreten (siehe Abschnitt~\ref{6.3}).
Aus Zeitgründen konnten praxistaugliche Lösungen hierfür nicht mehr als Teil dieser Arbeit entwickelt werden.

Die praktische Anwendbarkeit der entwickelten Algorithmen ist deshalb eingeschränkt auf Situationen mit nur wenigen oder einfachen Kreuzungen.

Im Vergleich zu den in Abschnitt~\ref{2.5} diskutierten existierenden Ansätzen zeigt sich ...
% - vgl. Schluss von 4.1: Idee der möglichst kurzen Linienstücke funktioniert grundsätzlich (jedoch: Effizienz fraglich)
% - ansatz geometrie funktioniert wohl im prinzip mehr oder weniger und ist neben den existierenden ansätzen nicht von der hand zu weisen, obwohl er aus zeitgründen noch nicht praxistauglich ist

...



\begin{itemize}
	\item abschließende qualitative Gesamtbeurteilung der Arbeit auf Basis der Ergebnisuntersuchung in Bezug auf:
	\begin{itemize}
		\item Praxistauglichkeit
		\item Übertragbarkeit auf andere als die spezifizierten Spezialfälle
		\item Übertragbarkeit auf andere, ähnlich gelagerte, aber nicht identische Fragestellungen (z. B. Generalisierung durch Verdrängen)
		\item evtl. in Relation zu existierenden Lösungsansätzen (-> Analyse)
		\item Ungelöste Problemfälle:
		\begin{itemize}
			\item gründlichere Untersuchung/Quantifizierung der „Praxistauglichkeit“
			\item ...
		\end{itemize}
	\end{itemize}
\end{itemize}
% - ansatz geometrie funktioniert wohl im prinzip mehr oder weniger und ist neben den existierenden ansätzen nicht von der hand zu weisen, obwohl er aus zeitgründen noch nicht praxistauglich ist



\section{Möglichkeiten zur Weiterentwicklung \emph{[Entwurf]}}



\subsection{Berücksichtigung unterschiedlicher Straßentypen}
% vgl. 6.1

\begin{itemize}

\item
Wie schon in Abschnitt~\ref{ch:result-trivial} erläutert, unterscheiden sich die in in Abschnitt~\ref{ch:analyse-algorithm} für die Definition von \textproc{Parallel} benutzten baulichen Kriterien für Abstand und Winkel der Richtungsfahrbahnen je nach Straßentyp.
Es wäre deshalb sinnvoll, diese Kriterien je nach Klassifikation der Straßen in der \osm-Datenbank entsprechend zu variieren. Die Fehlerrate in der Erkennung würde so geringfügig sinken.

\item
Diese kleine Verbesserung konnte lediglich aus Zeitgründen nicht mehr als Teil dieser Arbeit umgesetzt werden.

\end{itemize}



\subsection{Statistische Auswertung}
% vgl. 6.2

\begin{itemize}

\item
Abschnitt~\ref{ch:result-tags} erläutert, dass die gewählte Art und Weise der „absoluten“ Berücksichtigung von Attributen nicht zufriedenstellend funktioniert.

\item
Statt eines harten Ausschlusskriteriums wie in Abschnitt~\ref{ch:analyse-algorithm} definiert scheint es sich anbieten, aus der Geometrie und den Attributen eine \emph{Wahrscheinlichkeit} abzuleiten, mit der die verglichenen Segmente insgesamt parallel sind, und als einziges \emph{hartes} Kriterium einen Schwellenwert für diese Wahrscheinlichkeit vorzusehen.

\item
Es könnten dann auch leicht weitere Attribute wie Straßenname oder die vertikale Ebene in angemessenem Umfang berücksichtigt werden, ohne dass einzelne Aspekte, die falsch attributiert sind, ein korrektes Ergebnis verhindern.

\item
Diese Idee erscheint vielversprechend, konnte jedoch aus Zeitgründen nicht mehr als Teil dieser Arbeit umgesetzt werden.

\end{itemize}

% idee für zukunft: zusätzliches PARALLEL-kriterium "entgegengerichtet", bedarf aber evtl. eines tag-parsens und ggf. umdrehen bei oneway=-1



\subsection{Kreuzungserkennung}
% vgl. 6.3

\begin{itemize}

\item
Das in Abschnitt~\ref{ch:relocateGeneralisedNodes} beschriebene Verfahren zur Schließung von Topologielücken arbeitet in vielen Fällen nicht zufriedenstellend. Eine verbesserte Suche nach einem „geeigneten“ Punkt wäre vermutlich \emph{relativ} einfach möglich.

\item
Die derzeitige Strategie ist es, immer die nahesten Stützpunkte des generalisierten Linienzugs (also die kürzesten verknüpften Punktezuordnungen) als „geeignet“ zu betrachten.
Dies sind jedoch in vielen Fällen eben gerade nicht die „geeigneten“ Punkte.

\item
Nötig wäre eigentlich eine Analyse des Graphen des Straßennetzes, um anhand der \emph{vorher} existierenden Verknüpfungen im Netz \emph{hinterher} die bei der Generalisierung entstandenen Lücken wirklich sinnvoll schließen zu können.
%Die bereits in Abschnitt~\ref{ch:graph-based} diskutierten auf Graphenanalysen basierenden Ansätze

\item
Bereits eine deutliche Verbesserung für manche Situationen wäre zu erwarten, wenn verhindert würde, dass bei der Schließung von Topologielücken mehrere Punkte auf dieselbe Position verschoben werden.
Wird dieser Fall erkannt, so könnte der zweite Punkt beispielsweise auf die zweitkürzeste verknüpfte Punktezuordnung verschoben werden.
Dies löst jedoch nicht alle denkbaren Probleme.
% (löst nicht die schrecklichen "Haken", die sich manchmal bilden)

\item
Dass Kreuzungen nicht automatisiert als solche erkannt werden, hat sich als das größte Problem der entwickelten Algorithmen herausgestellt (Abschnitt~\ref{ch:missing-junction-detection}).
Diesen Mangel teilt der vom Verfasser gewählte Ansatz mit vielen der in Abschnitt~\ref{ch:existing-approaches} besprochenen älteren Ansätze.

\item
Die Implementierung einer praxistauglichen Kreuzungserkennung wäre demnach die Voraussetzung für eine gelungene automatisierte Zusammenfassung paralleler Linienzüge.
Sie scheint die vielversprechendste Möglichkeit zur Weiterentwicklung zu sein.

\item
Um Kreuzungen zu erkennen, sind unterschiedliche Ansätze denkbar.
Beispielsweise lassen sich wahrscheinlich Kreuzungen erkennen durch eine graphentheoretische Analyse des Straßennetzes auf kürzestmögliche Zyklen mit anschließender geometrischer Auswertung:
Unterschreitet die Fläche des Zyklus eine bestimmte Mindestgröße, so handelt es sich bei dessen Knoten möglicherweise um Bestandteile einer einzigen Straßenkreuzung, die auf einen einzelnen Punkt zusammengefasst werden können.
% https://de.wikipedia.org/wiki/Zyklus_(Graphentheorie)
% "Masche"?
% Graphenanalyse (s.~u.)

\item Länge (Screenshot mit roten Puffern)

\item {[MM99]} (2.5.5)

\end{itemize}



\subsection{Allgemeine Anwendbarkeit}
% statt nur der gewählte Spezialfall
% vgl. 6.5

\begin{itemize}

\item
Um die Anwendung auf andere Spezialfälle als die in Abschnitt~\ref{ch:case-selection} ausgewählten Richtungsfahrbahnen zu ermöglichen, wäre eine Verallgemeinerung der entwickelten Algorithmen und ihrer Implementierung im \term{Combiner} nötig.

\item
Abschnitt~\ref{ch:result-other-cases} beschreibt zwei konkrete Beispiele für Spezialfälle, auf welche die Ergebnisse dieser Arbeit wahrscheinlich übertragbar sind.
Um Praxistauglichkeit für diese Fälle zu erzielen, müsste der \term{Combiner} es erlauben, die auf den Spezialfall der Richtungsfahrbahnen zugeschnittenen Module zur Laufzeit durch entsprechende andere Module zu ersetzen.
Diese Möglichkeit wurde bei der Implementierung ausdrücklich vorgesehen (Abschnitt~\ref{ch:impl-analyser}), ist jedoch noch nicht voll nutzbar, da dies im Rahmen dieser Arbeit nicht erforderlich war (Abschnitt~\ref{ch:impl-special-case}).
Außerdem wäre eine Lösung zur Kreuzungserkennung erforderlich, möglicherweise wiederum auf den jeweiligen Spezialfall zugeschnitten.

\item
Es ist jedoch nicht anzunehmen, dass die Anwendung der entwickelten Algorithmen in allen in Abschnitt~\ref{ch:case-comparison} aufgeführten Spezialfällen erfolgreich wäre.
Die Definition des \textproc{Fußpunkt}s in Abschnitt~\ref{ch:split-algorithm} bringt mit sich, dass gegenüberliegende Segmente nur genau dann überhaupt auf Parallelität analysiert werden, wenn sie sich wenigstens teilweise überlappen.
% TODO: stimmt dies überhaupt?
Bei anthropogenen Parallelen wie gerade den für diese Arbeit zu betrachtenden Richtungsfahrbahnen ist dies normalerweise der Fall.
Segmente von Linienzügen, die natürlich Gewachsenes darstellen, können hingegen -- bei präziser Erfassung, wie sie von \osm\ im Grundsatz angestrebt wird -- derart windschief zueinander liegen, dass eine Überlappung gegenüberliegender Segmente nicht mehr durchgehend gegeben ist.
Eine Übertragung des in dieser Arbeit entwickelten Ansatzes auf solche Spezialfälle wie die in Abschnitt~\ref{vegetation-case-desc} besprochenen Vegetationsgrenzen ist deshalb wahrscheinlich unmöglich.

\end{itemize}

% die ANALYSE-Kriterien müssten auch jeweils optimiert werden; für Richtungsfahrbahnen scheinen sie zu funktionieren, könnten aber evtl. noch optimiert werden
% Zick-Zack-Problem?



\subsection{Softwarequalität}

In der Aufgabenstellung war nicht die Entwicklung einer \emph{Softwarelösung,} sondern die Entwicklung von \emph{Algorithmen} gefordert.
Die Implementierung in Software musste erfolgen, um die Praxistauglichkeit der Algorithmen beurteilen zu können, jedoch stand dabei die Qualität der Software selbst nicht im Vordergrund.
% vgl. 4.4

In den Abschnitten~\ref{ch:impl-architecture} und~\ref{ch:impl-difficulties} waren bereits Entscheidungen diskutiert worden, die zugunsten einer schnelleren Implementierung qualitative Nachteile mit sich brachten.
Eine Weiterentwicklung der folgenden Aspekte kann die Verwendbarkeit des \term{Combiners} durch den Endnutzer verbessern und wäre deshalb wünschenswert:
%
\begin{itemize}

\item
Die Umwandlung von \osm-Daten in Shapefiles (vgl. Abschnitt~\ref{ch:impl-architecture}) ist zum Teil umständlich.
Es wäre wünschenswert, dass der \term{Combiner} die für \osm\ üblichen Formate XML oder PBF direkt verwenden kann.
% "Osmpbf is a Java/C library to read and write OpenStreetMap PBF files."
Für kleinere Bereiche wäre auch die Möglichkeit einer direkten Abfrage der \osm-Datenbank denkbar.
\cf[233]{RT09}

\item
Die Datenausgabe ist ebenfalls verbesserungsfähig.
Das Verketten aller Segmente (vgl. Abschnitt~\ref{ch:impl-generalisation}) sollte besser eine optionale Operation sein, da sie je nach der vorgesehenen Weiterverarbeitung womöglich nicht benötigt wird.
Durch das Verketten ist es zudem schwierig, \term{tags} aus den Eingangsdaten beizubehalten, da sich \term{tags} im Verlauf einer Straße oft ändern.
Deshalb sind bisher nur wenige Attribute überhaupt im Ergebnis enthalten.

\item
Entsprechend Abschnitt~\ref{ch:impl-architecture} wird derzeit als internes Koordinatensystem nur die UTM-Zone~32 verwendet, welche gut zu dem in Kapitel~\ref{ch:result} verwendeten Testdatensatz passt.
% in io.Projection
Um Geodaten aus anderen Regionen der Welt verarbeiten zu können, muss der \term{Combiner} für eine andere UTM-Zone neu kompiliert werden.
Die jeweils am besten passende UTM-Zone ließe sich jedoch auch automatisch anhand der Eingangsdaten ermitteln.
% in io.ShapeReader

\item Effizienz / Parallelisierbarkeit
% vgl. 6.4 + 5.4.3

\begin{itemize}[nosep]
\item
...
\item
Teile der Algorithmen eignen sich für nebenläufige Ausführung.
Mit entsprechenden Anpassungen könnte die Leistungsfähigkeit moderner Hardware besser ausgenutzt werden.
% https://de.wikipedia.org/wiki/Nebenl%C3%A4ufigkeit
\end{itemize}

\item
Die Bedienung über die Kommandozeile (CLI) ist zwar ausreichend für das Testen der Algorithmen durch den Entwickler, ist jedoch für den Endnutzer unnötig anspruchsvoll.
\cf[363]{CR03}
Eine graphische Benutzeroberfläche (GUI) würde die Einstiegshürde erheblich senken.
Mit einer Integration in JOSM als \term{plug-in} (vgl. Abschnitt~\ref{ch:impl-architecture}) würden sich gleichzeitig die zuvor beschriebenen Probleme beim Einlesen der Ausgangsdaten lösen lassen, da JOSM alle angesprochenen Formate bereits unterstützt.

\end{itemize}

Des Weiteren ist gegenwärtig die Wiederverwendbarkeit des \term{Combiners} und seiner Bestandteile für Entwickler nicht optimal.
Verbesserungen hieran können sich indirekt auch auf den Endnutzer auswirken, indem sie die Entwicklung von Software mit weniger Fehlern fördern:
%
\begin{itemize}

\item Test Coverage
\begin{itemize}[nosep]
\item
praktisch null
\end{itemize}

\item Debugging Output
\begin{itemize}[nosep]
\item
vielerorts im Quellcode für Debugging-Output Kapselung verletzt etc.
\item
insb. \texttt{Dataset}  (s. „Big Renaming \#4“)
\item
evtl. common sink/listener
\end{itemize}

\item API-Schnittstellen für JTS/GeoTools
\begin{itemize}[nosep]
\item würde die Verwendung in anderen Projekten erleichtern
\item Abschnitt~\ref{ch:data-structures-geotools}
\end{itemize}

\item API-Dokumentation
\begin{itemize}[nosep]
\item noch Lückenhaft
\end{itemize}

\end{itemize}

% TODO: Big Renaming #4 (optional)
%1: de.thaw.thesis.comb -> de.thaw.comb
% - ShapeTagsAdapter -> ShapeGeofabrikAdapter
%2: Dataset
% special uses (to be reconsidered):
% - Output for debugging
% - Corr.Graph to get all segments
% - Gen.Lines to get all segments and to get midPoints
% - AbstractSegment for debugging (parallelFragment sink etc.)
%3: code currently contains a lot of special-cases for debugging as well as testing:
% - pervasive debug output needs to be overhauled; perhaps as a common sink/listener



\subsection{OSM Inspector}
% vgl. insb. 6.2

\begin{itemize}

\item
Der \term{OSM~Inspector} ist ein von der Geofabrik entwickeltes Web-Werkzeug, das Beitragenden helfen kann, bestimmte Fehler in der \osm-Datenbank zu entdecken.
\cf[142]{RT09}
Auch einige Attribute von Linienzügen im \osm-Straßennetz können damit überprüft werden:
„Highways [...] are one of the most important features in OSM.
This view helps finding problems on highways such as missing names, unusal highway types or wrong oneway tags.“
\citex{osm:InspectorHighways}

\item
Bisher kann der \term{OSM~Inspector} für Straßen immer nur einen einzelnen \term{way} prüfen.
Zusammenhänge zwischen mehreren \term{ways}, die Teil derselben Straße sind, können nicht erkannt werden.
Der in Abschnitt~\ref{ch:result-tags} beschriebene Fehler widersprüchlicher Attribute zweier Richtungsfahrbahnen bleibt deshalb im \term{OSM~Inspector} unentdeckt.

\item
Es ist denkbar, dass die Integration einer geometrischen Prüfung auf Parallelität -- wie sie die in dieser Arbeit entwickelten Algorithmen bieten -- in den \term{OSM~Inspector} sinnvoll sein könnte.
Es besteht die Chance, dass viele der angesprochenen Fälle fehlerhafter \term{tags} so durch die \osm-Community beseitigt werden können.
Dies würde die Datenqualität in \osm\ insgesamt erhöhen und damit die automatisierte Weiterverarbeitung nicht nur mit dem \term{Combiner} vereinfachen.

\item
Allerdings ist der \term{Combiner} zur Integration in den \term{OSM~Inspector} gegenwärtig nicht gut geeignet.
% obige Korrektheit-Probleme, Performance (weltweit?), Java cs. C++ etc.
Es ist auch fraglich, ob die mit einer Integration verbundenen Kosten den potenziellen Nutzen rechtfertigen.
% weitere Tools erwähnen?
% https://wiki.openstreetmap.org/wiki/Quality_assurance#Error_detection_tools

\end{itemize}



\section{Andere Ansätze und neuere Forschung \emph{[ausstehend]}}

\begin{itemize}
\item weitere Idee: Graphenanalyse: kleinste Ringe finden und prüfen, ob der umschlossene Raum eine ausreichend schmale und lange Form hat, um als parallel gelten zu können [ähnlich Tho05]
\item {[Mat17]}
\end{itemize}

% Arbeit aus dem alten Standpunkt schreiben! Neuere Forschung etc. hier (und evtl. in Einleitung Kontext erklären) -DGD



% single-chapter commands
%\onlyinsubfile{\listoffigures}
%\onlyinsubfile{% UTF-8

\documentclass[../main/thesis.tex]{subfiles}
\begin{document}

% include works in bibliography that aren't cited anywhere in the document (for debugging)
\onlyinsubfile{\nocite{*}}


\defbibnote{thesisBibIntro}{\justify%
Die Literaturangaben sind alphabetisch nach dem Kürzel sortiert.
Das Kürzel wird gebildet aus den ersten drei Buchstaben des Nachnamens des Autors, bei mehreren Autoren aus jeweils den Anfangsbuchstaben der Nachnamen, bei Körperschaften aus einer mnemonisch gewählten Folge von Kleinbuchstaben; jeweils ergänzt durch die letzten beiden Ziffern des Jahres der Veröffentlichung.
\par
Um ein eventuelles Nachschlagen zu erleichtern, sind die Referenzen wo immer möglich durch Angabe von Orten ergänzt, an denen eine Kopie des jeweiligen Werks am 1.~März 2018
% gegen 22~Uhr
aufzufinden war.
In der PDF-Ausgabe dieses Dokuments sind die URLs Hyperlinks.
Die Signaturen beziehen sich auf die Bibliothek des Karlsruher Instituts für Technologie und deren Standort „Fachbibliothek HsKA“.
\bigskip}


\RaggedRight
\addtocontents{toc}{\medskip}
\newpage\addcontentsline{toc}{chapter}{Literaturverzeichnis}
\printbibliography[title=Literaturverzeichnis,prenote=thesisBibIntro]

\end{document}
}
\end{document}
