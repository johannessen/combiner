% UTF-8

% single-chapter commands
\documentclass[../main/thesis.tex]{subfiles}
\onlyinsubfile{\setcounter{chapter}{6}}  % single-chapter command
\begin{document}


\chapter{Schlussfolgerung und Ausblick \emph{[ausstehend]}}

\section{Praktische Anwendbarkeit}

\begin{itemize}
	\item abschließende qualitative Gesamtbeurteilung der Arbeit auf Basis der Ergebnisuntersuchung in Bezug auf:
	\begin{itemize}
		\item Praxistauglichkeit
		\item Übertragbarkeit auf andere als die spezifizierten Spezialfälle
		\item Übertragbarkeit auf andere, ähnlich gelagerte, aber nicht identische Fragestellungen (z. B. Generalisierung durch Verdrängen)
		\item evtl. in Relation zu existierenden Lösungsansätzen (-> Analyse)
		\item Ungelöste Problemfälle:
		\begin{itemize}
			\item gründlichere Untersuchung/Quantifizierung der „Praxistauglichkeit“
			\item ...
		\end{itemize}
	\end{itemize}
\end{itemize}
% - ansatz geometrie funktioniert wohl im prinzip mehr oder weniger und ist neben den existierenden ansätzen nicht von der hand zu weisen, obwohl er aus zeitgründen noch nicht praxistauglich ist



\section{Möglichkeiten zur Weiterentwicklung}



\subsection{Berücksichtigung unterschiedlicher Straßentypen}
% vgl. 6.1
...



\subsection{Statistische Auswertung}
% vgl. 6.2
...

%Statt eines harten Ausschlusskriteriums wie in Abschnitt~\ref{ch:analyse-algorithm} definiert scheint es sich anbieten, aus der Geometrie und den Attributen eine \emph{Wahrscheinlichkeit} abzuleiten, mit der die verglichenen Segmente insgesamt parallel sind, und als einziges \emph{hartes} Kriterium einen Schwellenwert für diese Wahrscheinlichkeit vorzusehen.
%Es könnten dann auch leicht weitere Attribute wie Straßenname oder die vertikale Ebene in angemessenem Umfang berücksichtigt werden, ohne dass einzelne Aspekte, die falsch attribuiert sind, ein korrektes Ergebnis verhindern.
%Diese Idee wurde aus Zeitgründen nicht mehr als Teil dieser Arbeit umgesetzt.

% idee für zukunft: zusätzliches PARALLEL-kriterium "entgegengerichtet", bedarf aber evtl. eines tag-parsens und ggf. umdrehen bei oneway=-1



\subsection{Kreuzungserkennung}
% vgl. 6.3
\begin{itemize}
\item Verbesserte Suche nach einem „geeigneten“ Punkt für die Schließung von Topologielücken
% ^ vmtl. *relativ* einfach möglich. Die Strategie, einfach die nahesten Punkte (= kürzesten Matches) als geeignet zu betrachten, ist sowieso Banane; das sind höchstens zufällig die richtigen!
% => Untersuchung der _vorher_ existierenden Verknüpfungen im Netzgraphen wäre _eigentlich_ nötig, um hinterher die Lücken wirklich _gut_ schließen zu können
% evtl. möglicher schneller Fix für _manche_ Situationen: gdw. beide nodes eines Segments auf denselben node verschoben werden, einen von beiden weiterschieben auf den _zweit_nahesten Match
% (löst nicht die schrecklichen "Haken", die sich manchmal bilden)
\item evtl. Kreuzungen erkennen durch Graphenanalyse (s.~u.) und Höchstgröße der Ringe: falls unterschritten, auf einen einzigen node zusammenfassen
\end{itemize}



\subsection{Allgemeine Anwendbarkeit}
% statt nur der gewählte Spezialfall
% vgl. 6.5
...
% die ANALYSE-Kriterien müssten auch jeweils optimiert werden; für Richtungsfahrbahnen scheinen sie zu funktionieren, könnten aber evtl. noch optimiert werden
% Zick-Zack-Problem?



\subsection{Softwarequalität}

\begin{itemize}
\item I/O-Schnittstellen für OSM, andere Geodaten-Formate, kein hart gecodetes CRS mehr, ...
% "Osmpbf is a Java/C library to read and write OpenStreetMap PBF files."
\item Effizienz / Parallelisierbarkeit
% https://de.wikipedia.org/wiki/Nebenl%C3%A4ufigkeit
% vgl. 6.4
\item GUI
\item Test Coverage
\item API-Schnittstellen für JTS/GeoTools
\item API-Dokumentation
\end{itemize}

% TODO: Big Renaming #4 (optional)
%1: de.thaw.thesis.comb -> de.thaw.comb
% - ShapeTagsAdapter -> ShapeGeofabrikAdapter
%2: Dataset
% special uses (to be reconsidered):
% - Output for debugging
% - Corr.Graph to get all segments
% - Gen.Lines to get all segments and to get midPoints
% - AbstractSegment for debugging (parallelFragment sink etc.)
%3: code currently contains a lot of special-cases for debugging as well as testing:
% - CombinerMain: iterations / combineLines2
% - MyAnalyser: evaluateTags
% - pervasive debug output needs to be overhauled; perhaps as a common sink/listener



\subsection{OSM Inspector}

...



\section{Andere Ansätze und neuere Forschung}

\begin{itemize}
\item weitere Idee: Graphenanalyse: kleinste Ringe finden und prüfen, ob der umschlossene Raum eine ausreichend schmale und lange Form hat, um als parallel gelten zu können [ähnlich Tho05]
\item {[Mat17]}
\end{itemize}

% Arbeit aus dem alten Standpunkt schreiben! Neuere Forschung etc. hier (und evtl. in Einleitung Kontext erklären) -DGD



% single-chapter commands
%\onlyinsubfile{\listoffigures}
%\onlyinsubfile{% global bibliography settings

\nocite{*}  % include works in bibliography that aren't cited anywhere in the document (for debugging)

%\setbibpreamble{%
\defbibnote{thesisBibIntro}{%
Die Literaturangaben sind alphabetisch nach den Nachnamen der Autoren sortiert.
Bei mehreren Autoren wird nach dem ersten Autor sortiert.\par%
Um ein eventuelles Nachschlagen zu erleichtern, sind die Referenzen wo immer möglich durch Hyperlinks in dicktengleicher Schrift zu Orten ergänzt, an denen eine Kopie des jeweiligen Werks in den Tagen von der Abgabe dieser Arbeit aufzufinden war.
\bigskip}

%\defbibheading{bibliography}{\chapter*{Literaturverzeichnis}\addcontentsline{toc}{chapter}{Literaturverzeichnis}}

\newpage\addcontentsline{toc}{chapter}{Literaturverzeichnis}
\printbibliography[title=Literaturverzeichnis,prenote=thesisBibIntro]
}
\end{document}
