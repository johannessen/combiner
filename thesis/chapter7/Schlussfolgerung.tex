% UTF-8

% single-chapter commands
\documentclass[../main/thesis.tex]{subfiles}
\onlyinsubfile{\setcounter{chapter}{6}}  % single-chapter command
\begin{document}


\chapter{Schlussfolgerung und Ausblick \emph{[ausstehend]}}

\section{Praktische Anwendbarkeit}

\begin{itemize}
	\item abschließende qualitative Gesamtbeurteilung der Arbeit auf Basis der Ergebnisuntersuchung in Bezug auf:
	\begin{itemize}
		\item Praxistauglichkeit
		\item Übertragbarkeit auf andere als die spezifizierten Spezialfälle
		\item Übertragbarkeit auf andere, ähnlich gelagerte, aber nicht identische Fragestellungen (z. B. Generalisierung durch Verdrängen)
		\item evtl. in Relation zu existierenden Lösungsansätzen (-> Analyse)
		\item Ungelöste Problemfälle:
		\begin{itemize}
			\item gründlichere Untersuchung/Quantifizierung der „Praxistauglichkeit“
			\item ...
		\end{itemize}
	\end{itemize}
\end{itemize}
% - ansatz geometrie funktioniert wohl im prinzip mehr oder weniger und ist neben den existierenden ansätzen nicht von der hand zu weisen, obwohl er aus zeitgründen noch nicht praxistauglich ist



\section{Möglichkeiten zur Weiterentwicklung}



\subsection{Berücksichtigung unterschiedlicher Straßentypen}
% vgl. 6.1
...



\subsection{Statistische Auswertung}
% vgl. 6.2
...

% idee für zukunft: zusätzliches PARALLEL-kriterium "entgegengerichtet", bedarf aber evtl. eines tag-parsens und ggf. umdrehen bei oneway=-1



\subsection{Kreuzungserkennung}
% vgl. 6.3.2
\begin{itemize}
\item ...
\item evtl. Kreuzungen erkennen durch Graphenanalyse (s.~u.) und Höchstgröße der Ringe: falls unterschritten, auf einen einzigen node zusammenfassen
\end{itemize}



\subsection{Allgemeine Anwendbarkeit}
% statt nur der gewählte Spezialfall
% vgl. 6.5
...
% die ANALYSE-Kriterien müssten auch jeweils optimiert werden; für Richtungsfahrbahnen scheinen sie zu funktionieren, könnten aber evtl. noch optimiert werden
% Zick-Zack-Problem?



\subsection{Softwarequalität}

\begin{itemize}
\item I/O-Schnittstellen für OSM, andere Geodaten-Formate, kein hart gecodetes CRS mehr, ...
\item Effizienz / Parallelisierbarkeit
% https://de.wikipedia.org/wiki/Nebenl%C3%A4ufigkeit
% vgl. 6.4
\item GUI
\item API-Schnittstellen für JTS/GeoTools
\item API-Dokumentation
\end{itemize}



\subsection{OSM Inspector}

...



\section{Andere Ansätze und neuere Forschung}

\begin{itemize}
\item weitere Idee: Graphenanalyse: kleinste Ringe finden und prüfen, ob der umschlossene Raum eine ausreichend schmale und lange Form hat, um als parallel gelten zu können [ähnlich Tho05]
\item {[Mat17]}
\end{itemize}

% Arbeit aus dem alten Standpunkt schreiben! Neuere Forschung etc. hier (und evtl. in Einleitung Kontext erklären) -DGD



% single-chapter commands
%\onlyinsubfile{\listoffigures}
%\onlyinsubfile{% UTF-8

\documentclass[../main/thesis.tex]{subfiles}
\begin{document}

% include works in bibliography that aren't cited anywhere in the document (for debugging)
\onlyinsubfile{\nocite{*}}


\defbibnote{thesisBibIntro}{\justify%
Die Literaturangaben sind alphabetisch nach dem Kürzel sortiert.
Das Kürzel wird gebildet aus den ersten drei Buchstaben des Nachnamens des Autors, bei mehreren Autoren aus jeweils den Anfangsbuchstaben der Nachnamen, bei Körperschaften aus einer mnemonisch gewählten Folge von Kleinbuchstaben; jeweils ergänzt durch die letzten beiden Ziffern des Jahres der Veröffentlichung.
\par
Um ein eventuelles Nachschlagen zu erleichtern, sind die Referenzen wo immer möglich durch Angabe von Orten ergänzt, an denen eine Kopie des jeweiligen Werks am 1.~März 2018
% gegen 22~Uhr
aufzufinden war.
In der PDF-Ausgabe dieses Dokuments sind die URLs Hyperlinks.
Die Signaturen beziehen sich auf die Bibliothek des Karlsruher Instituts für Technologie und deren Standort „Fachbibliothek HsKA“.
\bigskip}


\RaggedRight
\addtocontents{toc}{\medskip}
\newpage\addcontentsline{toc}{chapter}{Literaturverzeichnis}
\printbibliography[title=Literaturverzeichnis,prenote=thesisBibIntro]

\end{document}
}
\end{document}
