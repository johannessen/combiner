% UTF-8

% single-chapter commands
\documentclass[../main/thesis.tex]{subfiles}
\onlyinsubfile{\setcounter{chapter}{3}}  % single-chapter command
\begin{document}


\chapter{Algorithmen zur Generalisierung}

\section{Vorgehensweise}

...

% -> alte Unterlagen durchgehen

% aus der Erinnerung:
% damals sehr früh (noch vor Anmeldung) den Algorithmus in groben Zügen aufgestellt und implementiert
% Vorgehen war im Prinzip, das Problem graphisch/geometrisch anzugehen und auf Papier zu lösen, dann in Code zu übertragen
% anschließend nur noch (sehr umfangreiche) Verbessserungen vorgenommen, insbesondere zur Flexibilisierung (individuellere Analyse, unterschiedliche Testdaten, Spezialfälle)

% an dieser Stelle außerdem Überleitung (?) -- big picture: wie hängen die folgenden teile zusammen?
% -> lt. Themenblatt soll das Analyseergebnis auch separat von der Generalisierung zu verwenden sein!
% d.h. die Main-Klasse / Fassade muss gar nicht unbedingt hier beschrieben werden, das kann auch ein Implementierungsdetail sein (in welchem Fall es evtl. unter Kap. 5 beschrieben werden sollte, wozu aber Kap. 5 wohl etwas umorganisiert werden müsste)

% im CLI sieht's im Moment in etwa so aus:
% 1. create OsmDataset (als InputDataset-Instanz, via ShapeReader)
% 2. Combiner.run
% 3. output
% also eigentlich nichts, was algorithmisch einer besonderen Beschreibung bedarf


\section{Beschreibung der Algorithmen}

...

\subsection{Identifikation parallel verlaufender Linien-Fragmente}

Konzept (abstrakt):

1. nur Segmente betrachten (gerade Linienabschnitte, definiert durch zwei Punkte)

2. alle nahe beieinander liegenden Segmente auf Parallelität untersuchen

Die Segmente sind jedoch unterschiedlich lang und liegen teilweise etwas „verstreut“ im Raum, was die Untersuchung erschwert.
Deshalb werden die Segmente zunächst fragmentiert, indem benachbarte Segmente „geschickt“ weiter in kürzere Segmente unterteilt werden, so dass parallele Segmente immer ähnlich lang sind und einander gegenüberstehen.

Im Detail:

1. geometrische Indizierung (R-Tree) der Eingabedaten, um Suche nach nahen LineParts zu ermöglichen [regionalise]

2. $\forall$ LineParts: AbstractLinePart.splitCloseParallels, um gut vergleichbare Stücke zu erhalten (reentrant/rekursiv, d. h. neu erzeugte Fragmente werden bis zu einer Mindestgröße immer weiter aufgeteilt) [split]

3. $\forall$ LineParts: $\forall$ nahe Parallelen (laut Index): [analyse]

3.1 Vorprüfung (boolean)

3.2 Hauptprüfung (double)

3.3 best matches (links/rechts getrennt) speichern (keine Nachprüfung => falls das best match nicht passt, wird es trotzdem genommen, sofern nicht schon die vorprüfung die sache abgebrochen hat)

% graphisch erklären, was genau passiert


\subsection{Generalisierung durch Zusammenfassung}

% 1. new CorrelationGraph
% 2. new GeneralisedLines

...


\subsection{Verknüpfung von Linienfragmenten zu einem einzigen kontinuierlichen Linienzug}

...


% single-chapter commands
%\onlyinsubfile{\listoffigures} \onlyinsubfile{\listoftables}
%\onlyinsubfile{% global bibliography settings

\nocite{*}  % include works in bibliography that aren't cited anywhere in the document (for debugging)

\setbibpreamble{Die Literaturangaben sind alphabetisch nach den Nachnamen der Autoren sortiert. Bei mehreren Autoren wird nach dem ersten Autor sortiert.\par\bigskip\bigskip}

\bibliography{../references-papers,../references-manual}
%\bibliography{../references-manual}
}
\end{document}
