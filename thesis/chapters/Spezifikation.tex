\chapter{Spezifikation der zu untersuchenden Fälle}

\section[Vergleich verschiedener Problemfälle]{Vergleich verschiedener Problemfälle der automatisierten Linien-Generalisierung}

\begin{itemize}
	\item mehrgleisige Eisenbahnstrecken
	\item Richtungsfahrbahnen im Straßenraum mit baulicher Trennung
	\item parallele Fahrbahnen/Wege für unterschiedliche Verkehrsmittel/-arten
	\begin{itemize}
		\item z. B. zweibahnige Straße außerorts mit parallelem Zweirichtungsradweg
		\item z. B. Nebenfahrbahnen (Frontage Roads)
		\item z. B. straßenbündiger Bahnkörper (Straße mit Gleisen für Straßenbahn oder Street-Running)
		\item z. B. komplexe innerstädtische Straße mit mehreren parallelen getrennten Radwegen, Gehwegen, Richtungsfahrbahnen, Nebenfahrbahnen, besonderem Stadtbahn-Gleiskörper (mehrgleisig), Adressinterpolationen, PLZ-Grenze und unterirdischer Fernwärmeleitung
		\item z. B. Rampen für Anschlussstellen vom Typ SPUI (der "Fall Kriegsstraße") oder Diamant
	\end{itemize}
\end{itemize}

(Vergleich unter dem Gesichtspunkt der Eignung als "Spezialfall" für diese DA, vgl. Themenblatt)


\section{Auswahl der in dieser Arbeit zu behandelnden Spezialfälle}

\begin{itemize}
	\item Welcher konkrete Spezialfall wird in dieser Arbeit stellvertretend untersucht? Gibt es einen wesentlichen, die Auswahl entscheidenden Grund?
	\item Kann das Ergebnis dieser Arbeit auf andere Spezialfälle oder auf die Gesamtheit verallgemeinert werden? (kurze, grobe Abschätzung auf Basis der zuvor benannten Beispiele)
\end{itemize}

