\chapter{Spezifikation der zu untersuchenden Fälle}

% Plural oder Singular?
% (vgl. Themenblatt)

\begin{itemize}
	
	% Welcher konkrete Spezialfall wird in dieser Arbeit stellvertretend untersucht?
	% Gibt es einen wesentlichen, die Auswahl entscheidenden Grund?
	% =============================================================
	
	\item als zu untersuchender Fall gewählt:
	
	\item Richtungsfahrbahnen im Straßenraum mit baulicher Trennung
		(autobahnähnliche Straßen, zweibahnige innerstädtische oder Überlandstraßen)
	
	\item Grund: es ist zu erwarten, dass für Straßen das Tagging weitgehend passt
	\begin{itemize}
		\item da Tags für viele Details existieren und genutzt werden (Beispiele)
		\item da viele Details fürs Straßennetz in den Standard-Karten von OSM gerendert werden
		\item wegen der praktischen Relevanz für Karten und Anwendungen in ganz allgemeinen Fällen
			(-> Allgemeinbevölkerung / gemeiner Mapper)
	\end{itemize}
	
	% Kann das Ergebnis dieser Arbeit auf andere Spezialfälle oder auf die Gesamtheit verallgemeinert werden?
	% (kurze, grobe Abschätzung auf Basis der zuvor benannten Beispiele)
	% ==================================================================
	
	\item Ergebnis u. U. übertragbar auf den Fall von parallelen Fahrbahnen/Wege für unterschiedliche Verkehrsmittel/-arten
	\begin{itemize}
		\item z. B. zweibahnige Straße außerorts mit parallelem Zweirichtungsradweg
		\item z. B. Nebenfahrbahnen (Frontage Roads)
		\item z. B. straßenbündiger Bahnkörper (Straße mit Gleisen für Straßenbahn oder Street-Running)
		\item z. B. komplexe innerstädtische Straße mit mehreren parallelen getrennten Radwegen, Gehwegen, Richtungsfahrbahnen, Nebenfahrbahnen, besonderem Stadtbahn-Gleiskörper (mehrgleisig), Adressinterpolationen, PLZ-Grenze und unterirdischer Fernwärmeleitung
		\item z. B. Rampen für Anschlussstellen vom Typ SPUI (der "Fall Kriegsstraße") oder Diamant
	\end{itemize}
	
	\item Ergebnis u. U. übertragbar auf den Fall von mehrgleisigen Eisenbahnstrecken
	
\end{itemize}
