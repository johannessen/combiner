\chapter{Ergebnisuntersuchung im Anwendungskontext}

\section{Beurteilungskriterien}

\begin{itemize}
	\item allgemeine Kriterien zur qualitativen Kritik des Ergebnisses (= des entwickelten Algorithmus / der entwickelten Software) beschreiben
	\item falls möglich, quantitative Metriken für die Beurteilung definieren
	\item erwartete Ergebnisse ausgehend von Analyse und Spezifikation rekapitulieren
\end{itemize}


\section{Ergebnisbeurteilung}

\begin{itemize}
	\item Beurteilung anhand der zuvor beschriebenen Kriterien
	\item Diskussion von signifikanten Abweichungen des beobachteten vom erwarteten Ergebnis
	\item Untersuchung der Anwendung des Algorithmus auf unterschiedliche Beispiele (unterschiedliche Regionen der Welt etc.) der spezifizierten Spezialfälle sowie auf andere als diese spezifizierten Fälle (highways statt railways etc.)
	\item abschließende quantitative Gesamtbeurteilung des Ergebnisses
\end{itemize}
