\chapter{Schlussfolgerung und Ausblick}

\section{Praktische Anwendbarkeit}

\begin{itemize}
	\item abschließende qualitative Gesamtbeurteilung der Arbeit auf Basis der Ergebnisuntersuchung in Bezug auf:
	\begin{itemize}
		\item Praxistauglichkeit
		\item Übertragbarkeit auf andere als die spezifizierten Spezialfälle
		\item Übertragbarkeit auf andere, ähnlich gelagerte, aber nicht identische Fragestellungen (z. B. Generalisierung durch Verdrängen)
		\item evtl. in Relation zu existierenden Lösungsansätzen (-> Analyse)
	\end{itemize}
\end{itemize}


\section{Ungelöste Problemfälle}

\begin{itemize}
	\item vorliegende Algorithmen und vorliegende Software
\end{itemize}


\section{Mögliche Ansätze zur Weiterentwicklung}

\begin{itemize}
	\item nächste Schritte
	\item neue Probleme
\end{itemize}


% Arbeit aus dem alten Standpunkt schreiben! Neuere Forschung etc. hier (und evtl. in Einleitung Kontext erklären) -DGD
