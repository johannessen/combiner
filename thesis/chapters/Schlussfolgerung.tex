\chapter{Schlussfolgerung und Ausblick}

\section{Praktische Anwendbarkeit}

\begin{itemize}
	\item abschließende qualitative Gesamtbeurteilung der Arbeit auf Basis der Ergebnisuntersuchung in Bezug auf:
	\begin{itemize}
		\item Praxistauglichkeit
		\item Übertragbarkeit auf andere als die spezifizierten Spezialfälle
		\item Übertragbarkeit auf andere, ähnlich gelagerte, aber nicht identische Fragestellungen (z. B. Generalisierung durch Verdrängen)
		\item evtl. in Relation zu existierenden Lösungsansätzen (-> Analyse)
	\end{itemize}
\end{itemize}
% - ansatz geometrie funktioniert wohl im prinzip mehr oder weniger und ist neben den existierenden ansätzen nicht von der hand zu weisen, obwohl er aus zeitgründen noch nicht praxistauglich ist


\section{Ungelöste Problemfälle}

\begin{itemize}
	\item vorliegende Algorithmen und vorliegende Software
\end{itemize}


\section{Mögliche Ansätze zur Weiterentwicklung}

\begin{itemize}
	\item nächste Schritte
	\item Parallelisierbarkeit
	\item neue Probleme
\end{itemize}


% Arbeit aus dem alten Standpunkt schreiben! Neuere Forschung etc. hier (und evtl. in Einleitung Kontext erklären) -DGD

% idee für zukunft: zusätzliches PARALLEL-kriterium "entgegengerichtet", bedarf aber evtl. eines tag-parsens und ggf. umdrehen bei oneway=-1

% weitere Idee: Graphenanalyse: kleinste Ringe finden und prüfen, ob der umschlossene Raum eine ausreichend schmale und lange Form hat, um als parallel gelten zu können; Kreuzungen erkennen durch Mindestgröße (falls unterschritten, auf einen einzigen node zusammenfassen) [ähnlich Tho05]
