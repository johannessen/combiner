% UTF-8

% single-chapter commands
\documentclass[../main/thesis.tex]{subfiles}
\begin{document}


\addtocontents{toc}{\bigskip\textbf{\protect\numberline{A}{Mathematische Konventionen}}\par}
% 4 (vgl. [Sar06])

\addtocontents{toc}{\bigskip\textbf{\protect\numberline{B}{Datenmodell und Klassenstruktur}}\par}
% 5.3.2 (ein Diagramm -- nur die zum eigentlichen Datenmodell gehörende Struktur)

\addtocontents{toc}{\bigskip\textbf{\protect\numberline{C}{Bezeichner im Quelltext}}\par}
% 5.3.6 (Kap. 4 vs. Source)

\addtocontents{toc}{\bigskip\textbf{\protect\numberline{D}{Anwendung auf ein Fernstraßennetz}}\par}
\label{appx:fullpage-examples-1}
% 6 (Beispiel für größeres Gebiet im Zusammenhang in kleinem Maßstab)
% evtl. unterschiedliche Regionen der Welt

\addtocontents{toc}{\bigskip\textbf{\protect\numberline{E}{Anwendung auf ein innerstädtisches Straßennetz}}\par}
\label{appx:fullpage-examples-2}
% 6 (Beispiel für größeres Gebiet im Zusammenhang in großem Maßstab)
% evtl. unterschiedliche Regionen der Welt

\addtocontents{toc}{\bigskip\textbf{\protect\numberline{F}{Beispiele für problematische Kreuzungssituationen}}\par}
\label{appx:junction-examples}
% 6 (weitere Beispiele des Scheiterns an unterschiedlichen Kreuzungen; auch zeigen, wie links das Topologielückenproblem hierbei nicht lösen, sondern vergrößern)

%\chapter{Glossar}
%\chapter{Abkürzungsverzeichnis}
%\chapter{Software-Dokumentation}


% single-chapter commands
\end{document}
